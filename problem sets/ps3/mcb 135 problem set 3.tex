\documentclass{article}
\newsavebox{\oldepsilon}
\savebox{\oldepsilon}{\ensuremath{\epsilon}}
\usepackage[minionint,mathlf,textlf]{MinionPro} % To gussy up a bit
\renewcommand*{\epsilon}{\usebox{\oldepsilon}}
\usepackage[margin=1in]{geometry}
\usepackage{graphicx} % For .eps inclusion
%\usepackage{indentfirst} % Controls indentation
\usepackage[compact]{titlesec} % For regulating spacing before section titles
\usepackage{adjustbox} % For vertically-aligned side-by-side minipages
\usepackage{array, amsmath,  mhchem}
\usepackage[hidelinks]{hyperref}
\usepackage{courier, subcaption}
\usepackage{multirow, enumerate}

\usepackage{float}
\restylefloat{table}

\pagenumbering{gobble} 
\setlength\parindent{0 cm}
\renewcommand{\arraystretch}{1.2}
\begin{document}
\large

MCB 135 Problem Set 3 \hfill Due Monday, February 23, 2015 at 2:30 PM

\section*{Problem 1: Transcription Factor Cascade (adapted from Alon 2.4, 30 points)}

Consider a cascade of three transcriptional activators, $X \rightarrow Y \rightarrow Z$. Protein $X$ is initially present in the cell in its inactive form. At time $t=0$, $X$ rapidly becomes active and binds the promoter of $Y$, so that protein $Y$ starts to be produced at a rate $\beta_y$. When $Y$ levels exceed a threshold $K_y$, gene $Z$ begins to be expressed at a rate $\beta_z$. All proteins have the same degradation/dilution rate $\alpha$.

\begin{enumerate}[a)]
\setlength{\itemsep}{0pt}
\item Using the information above, find an expression for the time derivative of $Y$. Integrate to find $Y(t)$ given the initial condition $Y(0)=0$. (Hint: A similar derivation is given in Alon Ch. 2.)
\item At what time $\tau$ does expression of $Z$ begin?
\item What is the concentration of protein $Z$ as a function of time? 
\item What is the \textit{response time} of the system, i.e., the time between when $X$ is activated and when $Z$ reaches one half of its maximal value?
\end{enumerate}

\section*{Problem 2: Multistep Ultrasensitivity (Ingalls 6.8.7, 40 points)}

In a 2005 paper, Jeremy Gunawardena presented a straightforward analysis of multistep ultrasensitivity and revealed that it is not well-described by Hill functions\footnote{\href{http://www.pnas.org/content/102/41/14617.abstract}{Gunawardena, J. (2005). Multisite protein phosphorylation makes a good threshold but can be a poor switch. \textit{PNAS}}}.\\

Consider a protein $S$ that undergoes a sequential chain of $n$ phosphorylations, all of which are catalyzed by the same kinase $E$, but each of which requires a separate collision event. Let $S_k$ denote the protein with $k$ phosphate groups attached. Suppose the phosphatase $F$ acts in a similar multistep manner. The reaction scheme is then
\[ \ce{S_0 <=> S_1 <=>} \cdots  \ce{<=> S_n } \]

In steady state, the net reaction rate at each step is zero.

\begin{enumerate}[a)]
\setlength{\itemsep}{0pt}
\item Consider the first phosphorylation-dephosphorylation cycle. Expanding the steps in the catalytic mechanism, we have
\[ \ce{S_0 + E <=>[k_{1E}][k_{-1E}] ES_0 ->[k_{catE}] E + S_1} \hspace{1 cm} \textrm{ and } \hspace{1 cm} \ce{S_1 + F <=>[k_{1F}][k_{-1F}] FS_1 ->[k_{catF}] F + S_0} \]

Apply a rapid equilibrium assumption to the association reactions ($S_0 + E \leftrightarrow ES_0$ and \linebreak $S_1 + F \leftrightarrow FS_1$) to describe the reaction rates as
\[ k_{\textrm{catE}} \left[ ES_0 \right] = \frac{k_{\textrm{catE}}}{K_{ME}} \left[ S_0 \right] \left[ E \right] \hspace{1 cm} \textrm{ and } \hspace{ 1cm} k_{\textrm{catF}} \left[ FS_1 \right] = \frac{k_{\textrm{catF}}}{K_{MF}} \left[ S_1 \right] \left[ F \right] \]

where $K_{ME} = k_{-1E}/k_{1E}$, $K_{MF} = k_{-1F}/k_{1F}$, and $[E]$ and $[F]$ are the concentrations of free kinase and phosphatase.

\item Use the fact that at steady state the net phosphorylation-dephosphorylation rate is zero to arrive at the equation
\[ \frac{\left[ S_1 \right]}{\left[ S_0 \right]} = \lambda \frac{\left[ E \right]}{\left[ F \right]}, \hspace{ 1 cm} \textrm{ where } \lambda = \frac{k_{\textrm{catE}}}{k_{\textrm{catF}}} \frac{K_{MF}}{K_{ME}}\]

\item Suppose the kinetic constants are identical for each phosphorylation-dephosphorylation step. In that case, verify that
\[ \frac{\left[ S_2 \right]}{\left[ S_1 \right]} = \lambda \frac{\left[ E \right]}{\left[ F \right]}, \hspace{0.5 cm}  \textrm{ so } \hspace{0.5 cm} \frac{\left[ S_2 \right]}{\left[ S_0 \right]} = \lambda^2 \left( \frac{\left[ E \right]}{\left[ F \right] } \right)^2 \hspace{0.5 cm}  \textrm{ and more generally } \hspace{0.5 cm}  \frac{\left[ S_j \right]}{\left[ S_0 \right]} = \lambda^j  \left( \frac{\left[ E \right]}{\left[ F \right] } \right)^j \]

\item Use the result in part c to write the fraction of protein $S$ that is in the fully phosphorylated form as
\begin{eqnarray}
\frac{\left[ S_n \right]}{\left[ S_{\textrm{total}} \right]} = \frac{\left( \lambda u \right)^n}{1 + \lambda u + \left( \lambda u \right)^2 + \ldots + \left( \lambda u \right)^n} \label{eqn:prob1d}
\end{eqnarray}

where $u=[E]/[F]$ is the ratio of kinase to phosphatase concentrations. Hint: $[S_{\textrm{total}}] = [S_0] + [S_1] + [S_2] + \ldots + [S_n]$.

\item Use the ratio $[E]/[F]$ to approximate the ratio of total concentrations $[E_{\textrm{total}}]/[F_{\textrm{total}}]$, so that equation \ref{eqn:prob1d} describes the system's dose-response. Plot this function (vs. $u$) for various values of $n$. Take $\lambda = 1$ for simplicity. Use your plots to verify Gunawardena's conclusion that for high values of $n$, these dose-response curves exhibit a threshold (in this case, at $u=1$) but their behavior cannot be described as switch-like, as they show near-hyperbolic growth beyond the threshold, regardless of the values of $n$.
\end{enumerate}

\section*{Problem 3: Fixed Point Analysis (Adapted from Ingalls 4.8.3, 30 points)}

Consider the general linear system
\begin{eqnarray*}
\frac{dx}{dt} & = & ax + by\\
\frac{dy}{dt} & = & cx + dy
\end{eqnarray*}

Note that the steady state is $(x,y)=(0,0)$. Choose six sets of parameter values ($a$, $b$, $c$, and $d$) that yield the following behaviors:

\begin{enumerate}[a)]
\setlength{\itemsep}{0pt}
\item Stable node (real negative eigenvalues)
\item Stable spiral point (complex eigenvalues with negative real part)
\item Center (purely imaginary eigenvalues)
\item Unstable spiral point (complex eigenvalues with positive real part)
\item Unstable node (real positive eigenvalues)
\item Saddle point (real eigenvalues of different sign)
\end{enumerate}

In each case, prepare a phase portrait of the system, including a direction field and a few representative trajectories. (Note: Unlike the problem in Ingalls, we will not require that you include the nullclines.) Hint: if either of the off-diagonal entries in the Jacobian matrix are zero, then the eigenvalues are simply the entries on the diagonal.

\end{document}