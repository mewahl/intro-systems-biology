\documentclass{article}
\newsavebox{\oldepsilon}
\savebox{\oldepsilon}{\ensuremath{\epsilon}}
\usepackage[minionint,mathlf,textlf]{MinionPro} % To gussy up a bit
\renewcommand*{\epsilon}{\usebox{\oldepsilon}}
\usepackage[margin=1in]{geometry}
\usepackage{graphicx} % For .eps inclusion
%\usepackage{indentfirst} % Controls indentation
\usepackage[compact]{titlesec} % For regulating spacing before section titles
\usepackage{adjustbox} % For vertically-aligned side-by-side minipages
\usepackage{array, amsmath,  mhchem}
\usepackage{hyper ref}
\usepackage{courier, subcaption}
\usepackage{multirow, color}
\usepackage[autolinebreaks,framed,numbered]{mcode}

\usepackage{float}
\restylefloat{table}

\pagenumbering{gobble} 
\setlength\parindent{0 cm}
\renewcommand{\arraystretch}{1.2}
\begin{document}
\large

\section*{Problem 1}

Ultimately we would like to calculate the rate of product formation, which from the law of mass action is given by:

\[ \frac{d\left[ P \right]}{dt} = k_2 \left[ C_2 \right] - k_{-2} \left[ P \right] \left[ E \right] \]

Assume that $[C_2]$ is at steady state:

\[ 0 = \frac{d \left[ C_2 \right]}{dt} =  k_r \left[ C_1 \right] + k_{-2} \left[ P \right] \left[ E \right] - k_2 \left[ C_2 \right] = k_r \left[ C_1 \right]  -  \frac{d\left[ P \right]}{dt}   \implies  \frac{d\left[ P \right]}{dt}  = k_r \left[ C_1 \right] \]

Assume that $[C_1]$ is also at steady state:

\[ 0 = \frac{d \left[ C_1 \right]}{dt} =  k_1 \left[ S \right] \left[ E \right] -\left(  k_{-1} + k_r \right)  \left[ C_1 \right] \implies \left[ C_1 \right]  = \frac{k_1 \left[ S \right] \left[ E \right]}{k_{-1} + k_r} \]

Given this expression and the above result:

\begin{eqnarray}
 \frac{d\left[ P \right]}{dt}  = \frac{k_1 k_r \left[ S \right] \left[ E \right]}{k_{-1} + k_r} \label{eqn:prob1a} 
 \end{eqnarray}

The desired rate law expression is equivalent to this statement, but does not contain the variable $[E]$, suggesting we must substitute an equivalent expression for $[E]$. Noting that the enzyme moiety is conserved, we can see that:

\begin{eqnarray*}
\left[ E_{\textrm{tot}} \right]  & = &  \left[ E\right]  + \left[ C_1 \right] + \left[ C_2 \right]\\
 \left[ E\right] & = & \left[ E_{\textrm{tot}} \right] - \left[ C_1 \right] - \left[ C_2 \right]\\
 & = & \left[ E_{\textrm{tot}} \right] - \frac{k_1}{k_{-1} + k_r} \left[ S \right] \left[ E \right] - \left[ C_2 \right]
\end{eqnarray*}

To simplify further, we must find an expression for $[C_2]$. The statement that the product readily rebinds free enzyme suggests that it may be appropriate to assume a rapid equilibrium assumption for the right-most reversible reaction:

\begin{eqnarray*}
k_2 \left[ C_2 \right] & = & k_{-2} \left[ P \right] \left[ E \right]\\
\left[ C_2 \right] & = & \frac{k_{-2} }{k_2} \left[ P \right] \left[ E \right]
\end{eqnarray*}

where we have again defined $K_p$ for convenience. Plugging this expression for $[C_2]$ into our expression for $[E]$ derived from moiety conservation and rearranging, we get:

\begin{eqnarray*}
 \left[ E\right] & = & \left[ E_{\textrm{tot}} \right] - \frac{k_1}{k_{-1} + k_r} \left[ S \right] \left[ E \right] - K_p \left[ P \right] \left[ E \right]\\
 \left(1 + \frac{k_1 \left[ S \right]}{k_{-1} + k_r} + \frac{k_{-2} }{k_2}  \left[ P \right] \right) \left[ E \right]   & = & \left[ E_{\textrm{tot}} \right]\\
  \left(\frac{k_{-1}+k_r}{k_1} + \left[ S \right]+ \frac{k_{-2} \left(k_{-1}+k_r \right)}{k_1k_2}  \left[ P \right] \right) \left[ E \right]   & = & \frac{k_{-1}+k_r}{k_1} \left[ E_{\textrm{tot}} \right]\\
   \left( K_m + \left[ S \right] + \frac{K_m}{K_p} \left[ P \right] \right) \left[ E \right] & = & K_m \left[ E_{\textrm{tot}} \right]\\
 \left[ E \right] & = & \frac{K_m \left[ E_{\textrm{tot}} \right]}{\left[ S \right] + K_m \left(1 + \frac{\left[ P \right]}{K_p} \right)}
\end{eqnarray*}

where we have defined $K_m$ and $K_p$ for convenience. This expression for $[E]$ can be plugged into equation \ref{eqn:prob1a} to get:

\begin{eqnarray*}
 \frac{d\left[ P \right]}{dt}  & = & \frac{k_r \left[ S \right]}{K_m} \left( \frac{K_m \left[ E_{\textrm{tot}} \right]}{\left[ S \right] + K_m \left(1 + \frac{\left[ P \right]}{K_p}  \right)} \right)  = \frac{V_{\textrm{max}} \left[ S \right]}{\left[ S \right] + K_m \left(1 + \frac{\left[ P \right]}{K_p} \right)}
 \end{eqnarray*}
 
 \section*{Problem 2}

We illustrate two approaches: the first uses rapid equilibrium assumptions for binding, and the second uses quasi-steady-state assumptions as the problem statement hints.

\subsection*{Rapid equilibrium assumptions}

Enzyme moiety conservation gives the expression:

\[ \left[ E_{\textrm{tot}} \right] = \left[ E \right] + \left[ ES \right] + \left[ EI \right] + \left[ ESI \right] \]

To simplify this, we will assume that all substrate and inhibitor binding events are in rapid equilibrium:

\begin{eqnarray*}
k_1 \left[ E \right] \left[ S \right] = k_{-1} \left[ ES \right] & \implies & \left[ E \right]  = \frac{k_{-1} \left[ ES \right]}{k_1 \left[ S \right]}\\
k_3 \left[ E \right] \left[ I \right] = k_{-3} \left[ EI \right] & \implies & \left[ EI \right]  = \frac{k_3 \left[ I \right]}{k_{-3}} \left( \frac{k_{-1} \left[ ES \right]}{k_1 \left[ S \right]} \right) = \frac{k_{-1}k_3 \left[ I \right] \left[ ES \right]}{k_{-3}k_1 \left[ S \right]}\\
k_3 \left[ ES \right] \left[ I \right] = k_{-3} \left[ ESI \right] & \implies & \left[ ESI \right] = \frac{k_3 \left[ ES \right] \left[ I \right]}{k_{-3}}
\end{eqnarray*}

Plugging these equations into the moiety conservation statement allows us to find an expression for [ES]:

\begin{eqnarray*}
\left[ E_{\textrm{tot}} \right] & = & \left( \frac{k_{-1}}{k_1 \left[ S \right]} + 1 + \frac{k_{-1}k_3 \left[ I \right]}{k_1 k_{-3} \left[ S \right]} + \frac{k_3 \left[ I \right]}{k_{-3}}  \right) \left[ ES \right]\\
& = & \left( 1 + \frac{k_3 \left[ I \right]}{k_{-3}} \right)\left( 1 + \frac{k_{-1}}{k_{1}\left[ S \right]} \right) \left[ ES \right]\\
\left[ ES \right] & = & \frac{\left[ E_{\textrm{tot}} \right]}{\left( 1 + \frac{k_3 \left[ I \right]}{k_{-3}} \right)\left( 1 + \frac{k_{-1}}{k_{1}\left[ S \right]} \right)} = \frac{\left[ E_{\textrm{tot}} \right] \left[ S \right]}{\left( 1 + \frac{\left[ I \right]}{K_i} \right)\left( \left[ S \right] + K_m\right)}
\end{eqnarray*}

Notice that the definition of $K_m$ here is a bit unexpected: this is due to our use of rapid equilibrium rather than steady-state assumptions. Plugging this into the equation for rate of product formation, we get:
\begin{eqnarray*}
\frac{d \left[ P \right]}{dt} = k_2 \left[ ES \right] = \left( \frac{k_2 \left[ E_{\textrm{tot}} \right]}{1 + \frac{\left[ I \right]}{K_i}} \right) \frac{\left[S\right]}{\left[ S \right] + K_m} = \left( \frac{V_{\textrm{max}}}{1 + \frac{\left[ I \right]}{K_i}} \right) \frac{\left[S\right]}{\left[ S \right] + K_m}
\end{eqnarray*}

\subsection*{Quasi-steady-state assumptions}

The quasi-steady-state assumptions give three expressions:

\begin{eqnarray*}
\frac{d\left[ EI \right]}{dt} & = & k_{3} \left[ E\right] \left[ I\right]  + k_{-1} \left[ ESI \right] - \left( k_1 \left[ S \right] + k_{-3} \right) \left[ EI \right] = 0 \\
\frac{d\left[ ES \right]}{dt} & = & k_{1} \left[ E\right] \left[ S \right]  + k_{-3} \left[ ESI \right] - \left( k_3 \left[ I \right] + k_{-1} + k_2 \right) \left[ ES \right] = 0\\
\frac{d\left[ ESI \right]}{dt} & = & k_{1} \left[ EI \right] \left[ S \right]  + k_{3} \left[ ES \right] \left[ I \right] - \left( k_{-3} + k_{-1} \right) \left[ ESI \right] = 0 \\
\end{eqnarray*}


We can eliminate [E] from these expressions using moiety conservation, i.e.

\[ \left[ E \right] = \left[ E_{\textrm{tot}} \right] - \left[ ES \right] - \left[ EI \right] - \left[ ESI \right] \]

Plugging in and rearranging, we get:

\begin{eqnarray*}
\frac{d\left[ EI \right]}{dt} & = & k_{3} \left[ E_{\textrm{tot}} \right] \left[ I\right]  - k_{3} \left[ I\right]  \left[ ES \right]  + \left(k_{-1} - k_3 \left[ I \right] \right) \left[ ESI \right] - \left( k_1 \left[ S \right] + k_{-3} + k_3 \left[ I \right] \right) \left[ EI \right] = 0 \\
\frac{d\left[ ES \right]}{dt} & = & k_{1} \left[ E_{\textrm{tot}} \right] \left[ S \right]  - k_{1} \left[ S \right] \left[ EI \right]  + \left( k_{-3} - k_1 \left[S\right] \right) \left[ ESI \right] - \left( k_3 \left[ I \right] + k_{-1} + k_2 + k_1 \left[S\right] \right) \left[ ES \right] = 0\\
\frac{d\left[ ESI \right]}{dt} & = & k_{1} \left[ EI \right] \left[ S \right]  + k_{3} \left[ ES \right] \left[ I \right] - \left( k_{-3} + k_{-1} \right) \left[ ESI \right] = 0 \\
\end{eqnarray*}

The third quasi-steady-state assumption gives us that:

\[ \left[ ESI \right] = \frac{k_{1}}{k_{-3} + k_{-1}} \left[ EI \right] \left[ S \right]  + \frac{k_{3}}{k_{-3} + k_{-1}} \left[ ES \right] \left[ I \right] \]

Plugging this in to the remaining two expressions, we get:

\begin{eqnarray*}
\frac{d\left[ EI \right]}{dt} & = & k_{3} \left[ E_{\textrm{tot}} \right] \left[ I\right]  - k_{3} \left[ I\right]  \left( \frac{k_{-3} + k_3 \left[ I \right]}{k_{-3} + k_{-1} } \right) \left[ ES \right]  - \left( k_1 \left[ S \right] \left( \frac{k_{-3} + k_3 \left[ I \right]}{k_{-3} + k_{-1} } \right) + k_{-3} + k_3 \left[ I \right] \right) \left[ EI \right] = 0 \\
\frac{d\left[ ES \right]}{dt} & = & k_{1} \left[ E_{\textrm{tot}} \right] \left[ S \right] 
- k_{1} \left[ S \right]  \left( \frac{k_{-1} + k_1 \left[ S \right]}{k_{-3} + k_{-1} } \right) \left[ EI \right] - \left( k_3 \left[ I \right] \left( \frac{k_{-1} + k_1 \left[ S \right]}{k_{-3} + k_{-1} } \right) + k_{-1} + k_2 + k_1 \left[S\right] \right) \left[ ES \right] = 0
\end{eqnarray*}

We can solve the first equation to find an expression for [EI], then plug this into the second equation and solve to find [ES]:

\begin{eqnarray*}
\left[ ES \right] & = & \frac{}{k_3 \left[ I \right] \left( \frac{k_{-1} + k_1 \left[ S \right]}{k_{-3} + k_{-1} } \right) + k_{-1} + k_2 + k_1 \left[S\right]}

\end{eqnarray*}


\begin{lstlisting}
function []  = mutualrepression()
    % Pick some parameter values for plotting
    global k n
    k = 0.5; n=3;
    
    [x, y] = meshgrid(0:0.05:1.5, 0:0.05:1.5);
    dx = k ./(k+y.^n) - x;
    dy = k ./(k+x.^n) - y;
    r = (dx.^2 + dy.^2).^0.5;
    dx = dx ./ r;
    dy = dy ./ r;
    
    quiver(x,y,dx,dy); hold on;
    xlabel('[X]')
    ylabel('[Y]')
    axis([0,1.5,0,1.5])
    [t, c] = ode45(@updater, [0 50], [0.7, 0.8]);
    plot(c(:,1),c(:,2),'-r', 'LineWidth', 3)
    plot(0.7, 0.8, 'or');
    [t, c] = ode45(@updater, [0 50], [0.6,0.5]);
    plot(c(:,1),c(:,2),'-g', 'LineWidth', 3);
    plot(0.6, 0.5, 'og');
    [t, c] = ode45(@updater, [0 50], [1.4,0.2]);
    plot(c(:,1),c(:,2),'-b', 'LineWidth', 3);
    plot(1.4, 0.2, 'ob');
    [t, c] = ode45(@updater, [0 50], [0.3,1.2]);
    plot(c(:,1),c(:,2),'-k', 'LineWidth', 3);
    plot(0.3, 1.2, 'ok');
    [t, c] = ode45(@updater, [0 50], [0.3,0.3]);
    plot(c(:,1),c(:,2),'-m', 'LineWidth', 3);
    plot(0.3, 0.3, 'om');
      
end

function dc = updater(t, c)
    x = c(1);
    y = c(2);
    global k n
    dx = k/(k+y^n) - x;
    dy = k/(k+x^n) - y;
    dc = [dx; dy];
end
\end{lstlisting}


\end{document}