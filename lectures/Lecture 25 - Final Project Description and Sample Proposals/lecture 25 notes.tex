\documentclass{article}
\usepackage[minionint,mathlf,textlf]{MinionPro} % To gussy up a bit
\usepackage[margin=1in]{geometry}
\usepackage{graphicx} % For .eps inclusion
%\usepackage{indentfirst} % Controls indentation
\usepackage[compact]{titlesec} % For regulating spacing before section titles
\usepackage{adjustbox} % For vertically-aligned side-by-side minipages
\usepackage{array, mathrsfs, mathrsfs, mhchem, amsmath} % For centering of tabulars with text-wrapping columns
\usepackage{hyperref, chemfig}
\usepackage{subfigure}
\newcommand{\Lapl}{\mathscr{L}}

\pagenumbering{gobble} 
\setlength\parindent{0 cm}
\begin{document}
\large

\section*{Final project}

\subsection*{What is the format of the research proposal?}

This assignment is modeled after the research proposal required for NSF Graduate Research Fellowship applications. It is limited to two single-spaced pages, including references and, if needed, a small figure. The proposal typically includes one or two introductory paragraphs, a list of 2-3 ``specific aims" (research objectives), 1-3 paragraphs describing how each aim will be accomplished, and possibly a separate paragraph describing broader impacts (see review criteria described below). This is not a long assignment, but there is an expectation that both the content and the writing will be highly polished.

\subsection*{Why should you practice writing research proposals?}

Regardless of the profession you choose -- and particularly when choosing your own profession! -- you must determine what interests you, decide what is most worthy of your time, and convince others to pay you to do it. Research proposals practice these skills in a framework that also demonstrated whether you have gleaned experimental design and interpretation skills practiced in this course. It is an opportunity for you to study a topic that excited you in more depth, brainstorm a plan for a summer project or thesis, and discuss your ideas with faculty who might become mentors.\\

Research proposals also have very direct and practical benefits for those considering graduate school and careers in research. You may already need a research proposal to apply for summer research stipends or to apply for graduate fellowships during your senior year. Ph.D. candidacy/qualifying exams in the sciences typically require the preparation and defense of a proposal. Competitive fellowships and grants often require research proposals as well as transcripts and test scores (for early career scientists), publications, personal statements, and letters of recommendation; because these grants and fellowships reflect many indications of aptitude, they become credentials in their own right, helping their recipients garner further support. While competitive grants and fellowships are not required to work at the graduate and postdoctoral levels, their recipients may receive more and better job offers. Applying for funding is of course a critical skill for success as a principal investigator, where the new submission approval rates are now below 20\% for both NIH (R01) and NSF (CAREER) grants.

\subsection*{What are the research proposal review criteria?}

Exercising the philanthropist's prerogative, funding agencies use research support to promote social as well as scientific agendas. The National Science Foundation gets its money from American taxpayers, and in turn represents their interests by incentivizing researchers to engage in education, training, service, and/or outreach to the general public. The NSF achieves this by incorporating an expectation to perform such activities into the criteria on which research proposals are judged. This is reflected in the NSF's two funding criteria:

\begin{itemize}
\item The Intellectual Merit criterion encompasses the potential to advance knowledge.
\item The Broader Impacts criterion encompasses the potential to benefit society and contribute to the achievement of specific, desired societal outcomes.
\end{itemize}

\subsection*{How much space should I devote to each?}

A good rule of thumb is to devote about the equivalent of a paragraph to broader impacts in a two-page research proposal. This may take the form of 3-5 sentences spread throughout the proposal that deal with broader impacts in context. (Lumping broader impacts together into a final paragraph may seem artless at first, but has the advantage that reviewers will not overlook them.)

\subsection*{What are some factors considered for the Intellectual Merit criterion?}

\begin{itemize}
\item Does the author demonstrate an understanding of the current state of relevant research?
\item Does the author motivate their choice of research question and approach?
\item What expertise does the author bring to the table? In whose lab would he or she work, and what are the existing resources and expertise there?
\item What experimental methods will the author employ? How will the results be interpreted, and what are the relevant controls or sanity checks?
\item Are other researchers studying related problems? How does this approach complement their work?
\item Will the research give publishable results regardless of the outcome? If not, what are the odds of failure and the contingency plan?
\end{itemize}

\subsection*{What are some examples of activities that meet Broader Impacts criterion?}

\begin{itemize}
\item Advancing the participation of women, minorities, and persons with disabilities in science, technology, engineering, and mathematics (STEM) fields (e.g. tutoring and mentoring younger students)
\item Increasing scientific literacy and public engagement with STEM (e.g. presenting your research to the general public, volunteering at museums, aquariums, zoos, and science fairs)
\item Involving members of the public in assisting with your research (``crowdsourcing")
\item Developing a diverse, globally-competitive workforce (e.g. forming or participating in international collaborations; developing new products and participating in the tech transfer process; elaborating benefits of your research for industry)
\item Fostering relationships between academia, industry and/or public policy (translational research that impacts the healthcare industry, promoting non-academic career options and appropriate training within academia, outreach to the industry and the legislature)
\end{itemize}

\subsection*{What are some examples of successful research proposals in this format?}

You can find many examples by googling ``NSF GRFP sample proposals," including these compilations:
\begin{itemize}
\item \href{https://biology.mit.edu/sites/default/files/NSF_proposals_102309.pdf}{MIT Biology}
\item \href{http://www.alexhunterlang.com/nsf-fellowship}{Suggestions and examples from various fields}, some including review feedback
\item The MCB Department maintains a folder of successful proposals written by our students outside Mike Lawrence's office in NWL 195.02.
\end{itemize}

\subsection*{What help is available for crafting the research proposal?}

I am happy to schedule a meeting in person or by Skype, or to provide feedback by email. (Please allow 2-3 days' turnaround time for me to review your draft.)\\

You may find that it is even more helpful to seek advice from an expert in the subfield that interests you: many faculty are excited to discuss research plans with students, and it's a great opportunity to meet a potential thesis adviser.\\

The \href{http://writingcenter.fas.harvard.edu/pages/undergraduates-0}{Harvard Writing Center} also offers one-on-one advice. You can ensure that you get someone with experience in scientific writing by scheduling an appointment or contacting the \href{http://writingcenter.fas.harvard.edu/pages/departmental-writing-fellows}{life sciences departmental writing fellow}.

\end{document}